\documentclass{uofa-eng-assignment}

\usepackage{lipsum}
\usepackage{dirtytalk} %quote
\usepackage{xcolor}

% Package For Circle
\usepackage{tikz}
\usetikzlibrary{arrows}
\usetikzlibrary{shapes}
\newcommand{\mymk}[1]{%
  \tikz[baseline=(char.base)]\node[anchor=south west, draw,rectangle, rounded corners, inner sep=2pt, minimum size=7mm,
    text height=2mm](char){\ensuremath{#1}} ;}

\newcommand*\circled[1]{\tikz[baseline=(char.base)]{
            \node[shape=circle,draw,inner sep=2pt] (char) {#1};}}
% Package for circle
        
\newcommand*{\name}{Wasis Haryo Sasoko}
\newcommand*{\id}{B0220068}
\newcommand*{\course}{Logika Informatika}
\newcommand*{\assignment}{Inferensi Logika}

\begin{document}

\maketitle

\begin{enumerate}

    %%%%%%%%%%%%%%%%%%%%
    \item Perlihatkan bahwa argumen berikut:

          \say{Jika air laut surut setelah gempa di laut, maka tsunami datang. Air laut surut setelah gempa dilaut. Karena itu tsunami datang.}

          adalah sahih\\\\
          %%%%%%%%%%%%%%%%%%%%
          \textbf{Jawab :}\\\\
          $p$ : Air laut surut setelah gempa di laut \\
          $q$ : Tsunami Datang \\


          $P_1$ : $p\xrightarrow{}q$ \\
          $P_2$ : p \\
          $K$ : q \\

          \begin{center}
              \begin{tabular}{c|c|c|c|c}
                  p & q & (p$\xrightarrow{}$q) & p           & q           \\
                  \hline
                  T & T & \circled{T}          & \circled{T} & \circled{T} \\
                  \hline
                  T & F & F                    & T           & F           \\
                  \hline
                  F & T & T                    & F           & T           \\
                  \hline
                  F & F & T                    & F           & F           \\
              \end{tabular}


              $\therefore$ Argumen adalah \textbf{Benar}
          \end{center}


          %%%%%%%%%%%%%%%%%%%%
    \item Perlihatkan bahwa penalaran pada argumen berikut: \\
          \say{Jika air laut surut setelah gempa di laut, maka tsunami datang. Tsunami datang. Jadi,
              air laut surut setelah gempa di laut} \\
          tidak benar, dengan kata lain argumennya palsu \\\\
          %%%%%%%%%%%%%%%%%%%%
          \textbf{Jawab :}\\\\
          $p$ : Air laut surut setelah gempa di laut \\
          $q$ : Tsunami Datang \\
          \\
          $P_1$ : $p\xrightarrow{}q$\\
          $P_2$ : $q$\\
          $K$ : $p$

          \begin{center}
              \begin{tabular}{c|c|c|c|c}
                  p & q & (p$\xrightarrow{}$q) & q           & p                            \\
                  \hline
                  T & T & \circled{T}          & \circled{T} & \circled{T}                  \\
                  \hline
                  T & F & F                    & F           & T                            \\
                  \hline
                  F & T & \circled{T}          & \circled{T} & \textcolor{red}{\circled{F}} \\
                  \hline
                  F & F & T                    & F           & F                            \\
              \end{tabular}


              $\therefore$ Argumen adalah \textbf{Salah}, karena tidak semua baris kritis bernilai Benar.
          \end{center}

          %%%%%%%%%%%%%%%%%%%%
    \item Periksa kesahihan argumen berikut ini:\\\\
          \begin{tabular}{l}
              Jika 5 lebih kecil dari 4, maka 5 bukan bilangan prima. \\
              5 tidak lebih kecil dari 4.                             \\
              \hline
              5 adalah bilangan prima                                 \\
          \end{tabular}\\\\
          %%%%%%%%%%%%%%%%%%%%
          \textbf{Jawab :}\\\\
          $p$ : 5 lebih kecil dari 4 \\
          $q$ : 5 bilangan prima \\
          \\
          $P_1$ : $p\xrightarrow{}\neg{q}$\\
          $P_2$ : $\neg{p}$\\
          $K$ : $q$ \\

          \begin{center}
              \begin{tabular}{c|c|c|c|c}
                  p & q & $p\xrightarrow{}\neg{q}$ & $\neg{p}$   & $q$                          \\
                  \hline
                  T & T & F                        & F           & T                            \\
                  \hline
                  T & F & T                        & F           & F                            \\
                  \hline
                  F & T & \circled{T}              & \circled{T} & \circled{T}                  \\
                  \hline
                  F & F & \circled{T}              & \circled{T} & \textcolor{red}{\circled{F}} \\
              \end{tabular}


              $\therefore$ Argumen adalah \textbf{Salah}, karena tidak semua baris kritis bernilai Benar.
          \end{center}


\end{enumerate}
\end{document}
