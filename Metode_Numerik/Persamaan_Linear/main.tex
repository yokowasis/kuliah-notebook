\documentclass[xcolor=table]{uofa-eng-assignment}

\usepackage{lipsum}
\usepackage{dirtytalk} %quote
\usepackage{xcolor}
\usepackage{enumitem}% http://ctan.org/pkg/enumitem
\usepackage{booktabs}% http://ctan.org/pkg/booktabs

% Package For Circle    
\usepackage{tikz}
\usetikzlibrary{arrows}
\usetikzlibrary{shapes}
\newcommand{\mymk}[1]{%
\tikz[baseline=(char.base)]\node[anchor=south west, draw,rectangle, rounded corners, inner sep=2pt, minimum size=7mm,
text height=2mm](char){\ensuremath{#1}} ;}

\newcommand*\circled[1]{\tikz[baseline=(char.base)]{
\node[shape=circle,draw,inner sep=2pt] (char) {#1};}}
% Package for circle

\newcommand*{\name}{Wasis Haryo Sasoko}
\newcommand*{\id}{B0220068}
\newcommand*{\course}{Metode Numerik}
\newcommand*{\assignment}{Persamaan Linear}

\begin{document}

\maketitle

\begin{enumerate}

    \item Selesaikan persamaan berikut :
          \begin{equation}
              x_1 + x_2 + 2x_3 = 4
          \end{equation}
          \begin{equation}
              2x_1 + 4x_2 - x_3 = 3
          \end{equation}
          \begin{equation}
              2x_1 + 3x_2 + x_3 = 10
          \end{equation}
          dengan Metode :
          \begin{enumerate}
              \item Eliminasi Gauss
              \item Eliminasi Gauss Jordan
              \item Iterasi Gauss Seidell
          \end{enumerate}
          \textbf{Jawab :}
          \medskip
          \begin{enumerate}
              \item Eliminasi Gauss \\
                    Augmented Matrix
                    \begin{equation}
                        \begin{bmatrix}
                            1 & 1 & 2  & 4  \\
                            2 & 4 & -1 & 3  \\
                            2 & 3 & 1  & 10
                        \end{bmatrix}
                    \end{equation}

                    \begin{enumerate}
                        \item Eliminasi\\
                              $B_2 = B_2 - 2B_1$ \\
                              $B_3 = B_3 - 2B_1$
                              \begin{equation}
                                  \begin{bmatrix}
                                      1 & 1 & 2  & 4  \\
                                      0 & 2 & -5 & -5 \\
                                      0 & 1 & -3 & 2
                                  \end{bmatrix}
                              \end{equation}
                        \item Eliminasi \\
                              $B_3 = 2B_3 - B2$
                              \begin{equation}
                                  \begin{bmatrix}
                                      1 & 1 & 2  & 4  \\
                                      0 & 2 & -5 & -5 \\
                                      0 & 0 & -1 & 9
                                  \end{bmatrix}
                              \end{equation}
                        \item Substitusi
                              \begin{equation}
                                  \begin{split}
                                      x_3 & = -9 \\
                                      x_2 & = \frac{-5+5(-9)}{2} \\
                                      & = \frac{-5-45}{2} \\
                                      & = \frac{-50}{2} \\
                                      & = -25 \\
                                      x_1 & = \frac{4-2(x_3)-1(x_2)}{1} \\
                                      & = 4-2(-9)-1(-25) \\
                                      & = 4+18+25 \\
                                      & = 47 \\
                                  \end{split}
                              \end{equation}
                        \item Hasil
                              \begin{equation}
                                  \begin{split}
                                      x_1 & = 47 \\
                                      x_2 & = -2 \\
                                      x_3 & = -9 \\
                                  \end{split}
                              \end{equation}
                    \end{enumerate}

              \item Elimasi Gauss Jordan \\
                    Augmented Matrix
                    \begin{equation}
                        \begin{bmatrix}
                            1 & 1 & 2  & 4  \\
                            2 & 4 & -1 & 3  \\
                            2 & 3 & 1  & 10
                        \end{bmatrix}
                    \end{equation}
                    \begin{enumerate}
                        \item Eliminasi\\
                              $B_2 = B_2 - 2B_1$ \\
                              $B_3 = B_3 - 2B_1$
                              \begin{equation}
                                  \begin{bmatrix}
                                      1 & 1 & 2  & 4  \\
                                      0 & 2 & -5 & -5 \\
                                      0 & 1 & -3 & 2
                                  \end{bmatrix}
                              \end{equation}
                        \item Eliminasi \\
                              $B_3 = 2B_3 - B2$
                              \begin{equation}
                                  \begin{bmatrix}
                                      1 & 1 & 2  & 4  \\
                                      0 & 2 & -5 & -5 \\
                                      0 & 0 & -1 & 9
                                  \end{bmatrix}
                              \end{equation}
                        \item Eliminasi \\
                              $B_1 = 2B_1 - B_2$
                              \begin{equation}
                                  \begin{bmatrix}
                                      2 & 0 & 9  & 13 \\
                                      0 & 2 & -5 & -5 \\
                                      0 & 0 & -1 & 9
                                  \end{bmatrix}
                              \end{equation}
                        \item Eliminasi \\
                              $B_1 = B_1 + 9B_3$
                              \begin{equation}
                                  \begin{bmatrix}
                                      2 & 0 & 0  & 94 \\
                                      0 & 2 & -5 & -5 \\
                                      0 & 0 & -1 & 9
                                  \end{bmatrix}
                              \end{equation}
                        \item Eliminasi \\
                              $B_2 = B_2 - 5B_3$
                              \begin{equation}
                                  \begin{bmatrix}
                                      2 & 0 & 0  & 94  \\
                                      0 & 2 & 0  & -50 \\
                                      0 & 0 & -1 & 9
                                  \end{bmatrix}
                              \end{equation}
                        \item Penyederhanaan \\
                              $B_1 = \frac{B_1}{2}$ \\
                              $B_2 = \frac{B_2}{2}$ \\
                              $B_3 = \frac{B_3}{-1}$
                              \begin{equation}
                                  \begin{bmatrix}
                                      1 & 0 & 0 & 47  \\
                                      0 & 1 & 0 & -25 \\
                                      0 & 0 & 1 & -9
                                  \end{bmatrix}
                              \end{equation}
                        \item Hasil
                              \begin{equation}
                                  \begin{split}
                                      x_1 & = 47 \\
                                      x_2 & = -25 \\
                                      x_3 & = -9 \\
                                  \end{split}
                              \end{equation}
                    \end{enumerate}
              \item Iterasi Gauss Seidell
                    \begin{equation}
                        \begin{split}
                            x_3 & = \frac{4-x_1-x_2}{2} \\
                            x_2 & = \frac{3-2x_1+x3}{4} \\
                            x_1 & = \frac{10-3x_2-x_3}{2} \\
                            e & = 0.0001 \\
                            x_{1_0} & = 0 \\
                            x_{2_0} & = 0 \\
                            x_{3_0} & = 0
                        \end{split}
                    \end{equation}
                    \begin{center}
                        $\begin{array}{c c c c c c c}
                                \toprule
                                i   & x_1         & x_2          & x_3          & ex_1     & ex_2      & ex_3      \\
                                \toprule
                                0   & 0           & 0            & 0            &          &           &           \\
                                1   & 5           & -1.75        & 0.375        & 5.000000 & -1.750000 & 0.375000  \\
                                2   & 7.4375      & -2.875       & -0.28125     & 2.437500 & -1.125000 & -0.656250 \\
                                3   & 9.453125    & -4.04688     & -0.70313     & 2.015625 & -1.171875 & -0.421875 \\
                                4   & 11.42188    & -5.13672     & -1.14258     & 1.968750 & -1.089844 & -0.439453 \\
                                5   & 13.27637    & -6.17383     & -1.55127     & 1.854492 & -1.037109 & -0.408691 \\
                                6   & 15.03638    & -7.15601     & -1.94019     & 1.760010 & -0.982178 & -0.388916 \\
                                7   & 16.7041     & -8.0871      & -2.3085      & 1.667725 & -0.931091 & -0.368317 \\
                                8   & 18.2849     & -8.96957     & -2.65766     & 1.580795 & -0.882477 & -0.349159 \\
                                9   & 19.78319    & -9.80601     & -2.98859     & 1.498295 & -0.836437 & -0.330929 \\
                                10  & 21.20331    & -10.5988     & -3.30225     & 1.420120 & -0.792792 & -0.313664 \\
                                11  & 22.54933    & -11.3502     & -3.59955     & 1.346020 & -0.751426 & -0.297297 \\
                                12  & 23.82512027 & -12.06244797 & -3.881336153 & 1.275788 & -0.712218 & -0.281785 \\
                                13  & 25.03434002 & -12.73750405 & -4.148417987 & 1.209220 & -0.675056 & -0.267082 \\
                                14  & 26.18046507 & -13.37733703 & -4.401564019 & 1.146125 & -0.639833 & -0.253146 \\
                                15  & 27.26678756 & -13.98378478 & -4.641501387 & 1.086322 & -0.606448 & -0.239937 \\
                                16  & 28.29642787 & -14.55858928 & -4.868919294 & 1.029640 & -0.574804 & -0.227418 \\
                                17  & 29.27234357 & -15.10340161 & -5.08447098  & 0.975916 & -0.544812 & -0.215552 \\
                                18  & 30.1973379  & -15.6197867  & -5.288775603 & 0.924994 & -0.516385 & -0.204305 \\
                                19  & 31.07406784 & -16.10922782 & -5.482420011 & 0.876730 & -0.489441 & -0.193644 \\
                                20  & 31.90505174 & -16.57313087 & -5.665960434 & 0.830984 & -0.463903 & -0.183540 \\
                                21  & 32.69267653 & -17.01282837 & -5.839924077 & 0.787625 & -0.439697 & -0.173964 \\
                                \multicolumn{7}{c}{...}                                                            \\
                                188 & 46.99814    & -24.999      & -8.99959     & 0.000102 & -0.000057 & -0.000023 \\
                                189 & 46.99824    & -24.999      & -8.99961     & 0.000097 & -0.000054 & -0.000021 \\
                                \toprule
                            \end{array}$
                    \end{center}
          \end{enumerate}
\end{enumerate}
\end{document}
