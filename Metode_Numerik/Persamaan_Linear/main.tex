\documentclass{uofa-eng-assignment}

\usepackage{lipsum}
\usepackage{dirtytalk} %quote
\usepackage{xcolor}

% Package For Circle
\usepackage{tikz}
\usetikzlibrary{arrows}
\usetikzlibrary{shapes}
\newcommand{\mymk}[1]{%
\tikz[baseline=(char.base)]\node[anchor=south west, draw,rectangle, rounded corners, inner sep=2pt, minimum size=7mm,
text height=2mm](char){\ensuremath{#1}} ;}

\newcommand*\circled[1]{\tikz[baseline=(char.base)]{
\node[shape=circle,draw,inner sep=2pt] (char) {#1};}}
% Package for circle

\newcommand*{\name}{Wasis Haryo Sasoko}
\newcommand*{\id}{B0220068}
\newcommand*{\course}{Metode Numerik}
\newcommand*{\assignment}{Persamaan Linear}

\begin{document}

\maketitle

\begin{enumerate}

    \item Selesaikan persamaan berikut : \\
          \begin{equation}
              x_1 + x_2 + 2x_3 = 4
          \end{equation}
          \begin{equation}
              2x_1 + 4x_2 - x_3 = 3
          \end{equation}
          \begin{equation}
              2x_1 + 3x_2 + x_3 = 10
          \end{equation}
          dengan Metode : \\
          \begin{enumerate}
              \item Eliminasi Gauss
              \item Eliminasi Gauss Jordan
              \item Iterasi
          \end{enumerate}
          \textbf{Jawab :}\\\\

          \begin{enumerate}
              \item Eliminasi Gauss \\
                    Augmented Matrix \\
                    \begin{equation}
                        \begin{bmatrix}
                            1 & 1 & 2  & 4  \\
                            2 & 4 & -1 & 3  \\
                            2 & 3 & 1  & 10
                        \end{bmatrix}
                    \end{equation}

                    \begin{enumerate}
                        \item Eliminasi \\
                              $B_2 = B_2 - 2B_1$ \\
                              $B_3 = B_3 - 2B_1$ \\
                              \begin{equation}
                                  \begin{bmatrix}
                                      1 & 1 & 2  & 4  \\
                                      0 & 2 & -5 & -5 \\
                                      0 & 1 & -3 & 2
                                  \end{bmatrix}
                              \end{equation}
                        \item Eliminasi \\
                              $B_3 = 2B_3 - B2$\\
                              \begin{equation}
                                  \begin{bmatrix}
                                      1 & 1 & 2  & 4  \\
                                      0 & 2 & -5 & -5 \\
                                      0 & 0 & -1 & 9
                                  \end{bmatrix}
                              \end{equation}
                        \item Substitusi \\
                              \begin{equation}
                                  \begin{split}
                                      x_3 & = -9 \\
                                      x_2 & = \frac{-5+5(-9)}{2} \\
                                      & = \frac{-5-45}{2} \\
                                      & = \frac{-50}{2} \\
                                      & = -25 \\
                                      x_1 & = \frac{4-2(x_3)-1(x_2)}{1} \\
                                      & = 4-2(-9)-1(-25) \\
                                      & = 4+18+25 \\
                                      & = 47 \\
                                  \end{split}
                              \end{equation}
                    \end{enumerate}

              \item Elimasi Gauss Jordan \\
              \item Iterasi \\
          \end{enumerate}


\end{enumerate}
\end{document}
